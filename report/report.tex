%%
\documentclass[format=acmlarge]{acmart}

%% \BibTeX command to typeset BibTeX logo in the docs
\AtBeginDocument{%
  \providecommand\BibTeX{{%
    Bib\TeX}}}

%% Rights management information.  This information is sent to you
%% when you complete the rights form.  These commands have SAMPLE
%% values in them; it is your responsibility as an author to replace
%% the commands and values with those provided to you when you
%% complete the rights form.
%% \setcopyright{acmlicensed}
%% \copyrightyear{2018}
%% \acmYear{2018}
%% \acmDOI{XXXXXXX.XXXXXXX}
%% These commands are for a PROCEEDINGS abstract or paper.
%% \acmConference[Conference acronym 'XX]{Make sure to enter the correct
%%   conference title from your rights confirmation email}{June 03--05,
%%   2018}{Woodstock, NY}
%%
%%  Uncomment \acmBooktitle if the title of the proceedings is different
%%  from ``Proceedings of ...''!
%%
%%\acmBooktitle{Woodstock '18: ACM Symposium on Neural Gaze Detection,
%%  June 03--05, 2018, Woodstock, NY}
%% \acmISBN{978-1-4503-XXXX-X/2018/06}

%%
%% end of the preamble, start of the body of the document source.
\begin{document}

%%
\title{Temporal Stability and Equilibrium Dynamics in Online Social Interactions: Case Study of a Subreddit Activity Network}

%%
\author{Maria João Vicente}
\email{fc44489@alunos.ciencias.ulisboa.pt}

\author{Pedro Fanica}
\email{fc54346@alunos.ciencias.ulisboa.pt}

\author{Quentin Weiss}
\email{fc66292@alunos.ciencias.ulisboa.pt}

%%
\begin{abstract}
  A clear and well-documented \LaTeX\ document is presented as an
  article formatted for publication by ACM in a conference proceedings
  or journal publication. Based on the ``acmart'' document class, this
  article presents and explains many of the common variations, as well
  as many of the formatting elements an author may use in the
  preparation of the documentation of their work.
\end{abstract}

%%
%% \keywords{}

%% A "teaser" image appears between the author and affiliation
%% information and the body of the document, and typically spans the
%% page.
%% \begin{teaserfigure}
%%   \includegraphics[width=\textwidth]{sampleteaser}
%%   \caption{Seattle Mariners at Spring Training, 2010.}
%%   \Description{Enjoying the baseball game from the third-base
%%   seats. Ichiro Suzuki preparing to bat.}
%%   \label{fig:teaser}
%% \end{teaserfigure}

%% \received{8 December 2025}
%% \received[revised]{12 March 2009}
%% \received[accepted]{5 June 2009}

%%
%% This command processes the author and affiliation and title
%% information and builds the first part of the formatted document.
\maketitle

\section{Introduction}

Online social communities generate complex patterns of user interaction
that evolve continuously over time. Platforms such as Reddit allow users
to participate through posts and replies, providing a rich source of
social data that can be modeled as networks and studied using techniques
from the domain of Network Science and Graph Theory. The analysis of such
networks is essential to understand how communities grow, stabilize,
fracture, or reorganize themselves.

In this work, we performed a longitudinal exam of the temporal
dynamics of a specific Subreddit (\verb|r/Documentaries|), between the years
2015 and 2018. To that end, we constructed periodical monthly snapshots
of this Subreddit's user interaction network. Each snapshot is a distinct
network, where users are represented as nodes and reply relationships
as weighted, undirected edges, capturing both the existence and intensity
of interactions.

By computing a suite of structural metrics, such as degree distribution,
clustering coefficients, centrality measures, communities, and component
structure, and plotting their trajectories across time, we aim to characterize
the evolution of user behavioral patterns, identify persistent structural
features and assess the overall stability of this Subreddit's underlying
social fabric.

The goal is to describe the temporal behaviour of this Subreddit, but also
to explore whether this online community exhibits some type of stable
signature in their network properties or whether, on the other hand, it
evolves unpredictably. To that end, we also created a theorical network
model and measured how it approximated the observed community.

\section{Related Work}

Code snippet example:
\begin{verbatim}
  \documentclass[STYLE]{acmart}
\end{verbatim}

Text can be in {\itshape italic style}, {\bfseries bold style},
\underline{underlined} or {\verb|code|}.

\section{Data}

\subsection{Data Source and Choice}

The base data for this case study was downloaded as a zip file from ConvoKit, the Cornell
Conversation Analysis Toolkit, at /url{https://zissou.infosci.cornell.edu/convokit/datasets/subreddit-corpus/corpus-zipped/}.
According to their documentation:

\say{This toolkit contains tools to extract conversational features and analyze social
phenomena in conversations, using a single unified interface inspired by (and compatible with)
scikit-learn. Several large conversational datasets are included together with scripts
exemplifying the use of the toolkit on these datasets.}

This website includes extracts from multiple Subreddits. Our criteria to select the
r/Documentaries Subreddit as the object of our study was:
\begin{enumerate}
  \item {\bfseries Size of the dataset:} at 340 MB, this seemed like a manageable dataset that
  could produce networks with a sufficient number of nodes to derive meaningful results
  from without compromising performance or operational agility. Our aim was to work with
  networks that contained at least 10 000 nodes.
  \item {\bfseries Reference in previous work:} The r/Documentaries Subreddit was mentioned in
  the 2018 article {\itshape Community Interaction and Conflict on the Web}, by S. Kumar, W. L.
  Hamilton, J. Leskovec, and D. Jurafsky\cite{Kumar2018}. In their study, the authors identified
  a group of users from another Subreddit, who mobilized together and initiated a targetted
  attack against r/Documentaries, by posting malicious responses to posts in that Subreddit.
  Given our intention to analyse the stability and cohesion of an online community over time,
  the existence of such conflicts that necessarily interfere with regular user activity was
  of particular interest to us.
\end{enumerate}

\subsection{Tabular Data Preparation}

The base dataset, in tabular format, included data from the time of the Subreddit's creation up
to October 2018. Some notes on the structure of the data and nomenclature:
\begin{itemize}
  \item {\bfseries Users} are identified by their Reddit username.
  \item An individual post or a comment is an {\bfseries utterance}.
  \item A post and its respective response comments is a {\bfseries conversation}.
  \item Users whose profile was deleted have their username replaced with the string {\verb|[deleted]|}.
  \item Deleted utterances have their content replaced with the string {\verb|[removed]|}.
\end{itemize}

Our data preprocessing pipeline started with the loading of the conversation and utterance datasets,
which we called {\bfseries posts} and {\bfseries comments}, respectively. We then performed a series
of preprocessing operations, namely:
\begin{itemize}
  \item Parsing conversation timestamps: The timestamp of the start of a conversation
  was converted from a unix-timestamp to a datetime object.
  \item Parsing utterance timestamps: The timestamp in which an utterance was produced
  was converted from a unix-timestamp to a datetime object. When the original timestamp
  could not be parsed, the records were dropped from the dataset. In the remaining records,
  we created four derived features based on the timestamp: the week of year number, the month
  of year number, the year, and the string concatenating year and month in YYYY-MM format.
  These new features enabled us to identify which utterances belongeded to each network snapshot
  in later preprocessing steps.
  \item Excluding invalid utterances based on user details: To prevent skewing the results,
  we excluded all utterances where the user was identified as {\verb|[deleted]|} or was left undefined.
  If these utterances had been left in the dataset, they would have produced user nodes in the networks
  with a substantially higher degree than the others, as they would have aggregated the activity from
  an unknown number of users. For the same reason, we also excluded all utterances where the reply-to
  user was undefined (note: the first utterance in a conversation is considered to be a self-reply,
  so this field should not be empty under normal circumstances).
  \item Excluding invalid utterances based on text: Our goal was to derive new attributes from the
  utterance text that could be used to characterize the network and communities found therein.
  As such, we excluded all utterances in which the text was unavailable.
  \item Calculating sentiment analysis score for each utterance: We used the VADER SentimentIntensityAnalyzer
  model to calculate the sentiment score for each utterance. This was an example of an attribute
  that could be extracted from the text of the utterances that could be used to characterize
  communities of users. Other attributes considered were a controlled subset of topics, but due
  to time constraints, these were not implemented.
\end{itemize}

\subsection{Network Preparation}

With the base data preprocessed and formatted, the next step in our process was to create user-user
networks based on the tabular utterance data. Our algorithm worked by taking as inputs the number of
snapshots per year to be created and the temporal granularity of the data (weekly or monthly). For
the purpose of the results presented in this report, we generated four monthly snapshot networks,
according to the following process:

\begin{enumerate}
  \item {\bfseries Size of the dataset:}
  \item {\bfseries Reference in previous work:}
\end{enumerate}

\section{Methods}

\subsection{Community Detection}
\subsection{Metrics}
\subsection{Theoretical Network Generation Models}

\section{Results and Discussion}

\subsection{Observed Subreddit Networks}
\subsection{Comparison with Theoretical Model}

\section{Conclusions}

\subsection{Subsection}
\label{sec:subsection}

This is a subsection.

\subsubsection{Subsubsection}
\label{sec:subsubsection}

This is a subsubsection.

\paragraph{Paragraph}

This is a paragraph.

\subparagraph{Subparagraph}

This is a subparagraph.

\section{Tables}

Immediately following this sentence is the point at which
Table~\ref{tab:freq} is included in the input file; compare the
placement of the table here with the table in the printed output of
this document.

\begin{table}
  \caption{Frequency of Special Characters}
  \label{tab:freq}
  \begin{tabular}{ccl}
    \toprule
    Non-English or Math&Frequency&Comments\\
    \midrule
    \O & 1 in 1,000& For Swedish names\\
    $\pi$ & 1 in 5& Common in math\\
    \$ & 4 in 5 & Used in business\\
    $\Psi^2_1$ & 1 in 40,000& Unexplained usage\\
  \bottomrule
\end{tabular}
\end{table}

Immediately following this sentence is the point at which
Table~\ref{tab:commands} is included in the input file.

\begin{table*}
  \caption{Some Typical Commands}
  \label{tab:commands}
  \begin{tabular}{ccl}
    \toprule
    Command &A Number & Comments\\
    \midrule
    \texttt{{\char'134}author} & 100& Author \\
    \texttt{{\char'134}table}& 300 & For tables\\
    \texttt{{\char'134}table*}& 400& For wider tables\\
    \bottomrule
  \end{tabular}
\end{table*}

\section{Figures}

The ``\verb|figure|'' environment should be used for figures. One or
more images can be placed within a figure. If your figure contains
third-party material, you must clearly identify it as such, as shown
in the example below.
\begin{figure}[h]
  \centering
  \includegraphics[width=\linewidth]{sample-franklin}
  \caption{1907 Franklin Model D roadster. Photograph by Harris \&
    Ewing, Inc. [Public domain], via Wikimedia
    Commons. (\url{https://goo.gl/VLCRBB}).}
  \Description{A woman and a girl in white dresses sit in an open car.}
\end{figure}

Your figures should contain a caption which describes the figure to
the reader.

Figure captions are placed {\itshape below} the figure.

Every figure should also have a figure description unless it is purely
decorative. These descriptions convey what’s in the image to someone
who cannot see it. They are also used by search engine crawlers for
indexing images, and when images cannot be loaded.

A figure description must be unformatted plain text less than 2000
characters long (including spaces).  {\bfseries Figure descriptions
  should not repeat the figure caption – their purpose is to capture
  important information that is not already provided in the caption or
  the main text of the paper.} For figures that convey important and
complex new information, a short text description may not be
adequate. More complex alternative descriptions can be placed in an
appendix and referenced in a short figure description. For example,
provide a data table capturing the information in a bar chart, or a
structured list representing a graph.  For additional information
regarding how best to write figure descriptions and why doing this is
so important, please see
\url{https://www.acm.org/publications/taps/describing-figures/}.

\section{Citations and Bibliographies}

The use of \BibTeX\ for the preparation and formatting of one's
references is strongly recommended. Authors' names should be complete
--- use full first names (``Donald E. Knuth'') not initials
(``D. E. Knuth'') --- and the salient identifying features of a
reference should be included: title, year, volume, number, pages,
article DOI, etc.

The bibliography is included in your source document with these two
commands, placed just before the \verb|\end{document}| command:
\begin{verbatim}
  \bibliographystyle{ACM-Reference-Format}
  \bibliography{bibfile}
\end{verbatim}
where ``\verb|bibfile|'' is the name, without the ``\verb|.bib|''
suffix, of the \BibTeX\ file.

Citations and references are numbered by default. A small number of
ACM publications have citations and references formatted in the
``author year'' style; for these exceptions, please include this
command in the {\bfseries preamble} (before the command
``\verb|\begin{document}|'') of your \LaTeX\ source:
\begin{verbatim}
  \citestyle{acmauthoryear}
\end{verbatim}


  Some examples.  A paginated journal article \cite{Abril07}, an
  enumerated journal article \cite{Cohen07}, a reference to an entire
  issue \cite{JCohen96}, a monograph (whole book) \cite{Kosiur01}, a
  monograph/whole book in a series (see 2a in spec. document)
  \cite{Harel79}, a divisible-book such as an anthology or compilation
  \cite{Editor00} followed by the same example, however we only output
  the series if the volume number is given \cite{Editor00a} (so
  Editor00a's series should NOT be present since it has no vol. no.),
  a chapter in a divisible book \cite{Spector90}, a chapter in a
  divisible book in a series \cite{Douglass98}, a multi-volume work as
  book \cite{Knuth97}, a couple of articles in a proceedings (of a
  conference, symposium, workshop for example) (paginated proceedings
  article) \cite{Andler79, Hagerup1993}, a proceedings article with
  all possible elements \cite{Smith10}, an example of an enumerated
  proceedings article \cite{VanGundy07}, an informally published work
  \cite{Harel78}, a couple of preprints \cite{Bornmann2019,
    AnzarootPBM14}, a doctoral dissertation \cite{Clarkson85}, a
  master's thesis: \cite{anisi03}, an online document / world wide web
  resource \cite{Thornburg01, Ablamowicz07, Poker06}, a video game
  (Case 1) \cite{Obama08} and (Case 2) \cite{Novak03} and \cite{Lee05}
  and (Case 3) a patent \cite{JoeScientist001}, work accepted for
  publication \cite{rous08}, 'YYYYb'-test for prolific author
  \cite{SaeediMEJ10} and \cite{SaeediJETC10}. Other cites might
  contain 'duplicate' DOI and URLs (some SIAM articles)
  \cite{Kirschmer:2010:AEI:1958016.1958018}. Boris / Barbara Beeton:
  multi-volume works as books \cite{MR781536} and \cite{MR781537}. A
  presentation~\cite{Reiser2014}. An article under
  review~\cite{Baggett2025}. A
  couple of citations with DOIs:
  \cite{2004:ITE:1009386.1010128,Kirschmer:2010:AEI:1958016.1958018}. Online
  citations: \cite{TUGInstmem, Thornburg01, CTANacmart}.
  Artifacts: \cite{R} and \cite{UMassCitations}.

%%
%% The acknowledgments section is defined using the "acks" environment
%% (and NOT an unnumbered section). This ensures the proper
%% identification of the section in the article metadata, and the
%% consistent spelling of the heading.
%% \begin{acks}
%% To Robert, for the bagels and explaining CMYK and color spaces.
%% \end{acks}

%%
%% The next two lines define the bibliography style to be used, and
%% the bibliography file.
\bibliographystyle{ACM-Reference-Format}
\bibliography{bibliography}


%%
%% If your work has an appendix, this is the place to put it.
\appendix

\section{Research Methods}

\subsection{Part One}

Lorem ipsum dolor sit amet, consectetur adipiscing elit. Morbi
malesuada, quam in pulvinar varius, metus nunc fermentum urna, id
sollicitudin purus odio sit amet enim. Aliquam ullamcorper eu ipsum
vel mollis. Curabitur quis dictum nisl. Phasellus vel semper risus, et
lacinia dolor. Integer ultricies commodo sem nec semper.

\subsection{Part Two}

Etiam commodo feugiat nisl pulvinar pellentesque. Etiam auctor sodales
ligula, non varius nibh pulvinar semper. Suspendisse nec lectus non
ipsum convallis congue hendrerit vitae sapien. Donec at laoreet
eros. Vivamus non purus placerat, scelerisque diam eu, cursus
ante. Etiam aliquam tortor auctor efficitur mattis.

\section{Online Resources}

Nam id fermentum dui. Suspendisse sagittis tortor a nulla mollis, in
pulvinar ex pretium. Sed interdum orci quis metus euismod, et sagittis
enim maximus. Vestibulum gravida massa ut felis suscipit
congue. Quisque mattis elit a risus ultrices commodo venenatis eget
dui. Etiam sagittis eleifend elementum.

\end{document}
\endinput
